\documentclass[a4paper,11pt,twocolumn]{article}

\usepackage[small,bf]{caption}
\usepackage{xltxtra}
\usepackage{fontspec}

\defaultfontfeatures{Scale=MatchLowercase,Mapping=tex-text}

\usepackage{times}
\usepackage{natbib}

\setromanfont{Times New Roman}
%\setmonofont{FreeMono}


\bibdata{paper}


\title{A North Sámi to South Sámi machine translation prototype}
\author{Lene Antonsen,\\Giellatekno,\\Tromsø\\{\tt lene.antonsen@uit.no}
\and Francis Tyers,\\Giellatekno,\\Tromsø\\{\tt ftyers@dlsi.ua.es}
\and Trond Trosterud,\\Giellatekno,\\Tromsø\\{\tt trond.trosterud@uit.no}}
\date{}

\begin{document}

\maketitle

\section{Introduction}
%@trond
%To be written after the paper is finished, but try to state the idea.

The paper presents a machine translation system from North Saami to
South Saami. The system is intended to work in a translation setting
where North Saami acts as a pivot language, with manual translation
from Norwegian into the largest of the Saami languages, and thereafter
to offer machine translation service, with postediting, to the other,
smaller, Saami languages. On the one hand, the Saami languages are
closely related, and therefore lend themselves better to MT than a
system translating from Norwegian directly, but on the other hand, the
classical problems of translating via a pivot language apply in this
case as well. 

The paper is structured as follows. Afther looking at previous work
and at the languages themselves, we give a presentation of the actual
machine translation system. Thereafter comes an evaluation of the
system. We presented translated text to translators, alongside with
the Norwegian original, without revealing the fact that the texts were
actually translated from North Saami.

finally comes a conclusion.


% Previous work
\cite{tyers09} \cite{wiechetek10} \cite{trosterud12}
% LREC?

%pivot translation references --babych paper

\section{Languages}
%@trond
% [Sámi language area map]

Norwegian North Germanic. The Saami languages constitute the
westernmost branch of the Uralic language family. They contain most of
the classical Uralic characteristica: A rich verbal morphology (3
numbers and persons, a rich repertoire of infinite forms), and a
medium-size case system with both grammatical and adverbial functions,
and no gender distinction. They also show extensive contact with their
Germanic neighbours. Many grammatical structures ahre head-initial
constructions, as compared to the classical Uralic pattern, and the
tense system is compatible with the North Germanic one.

There are also differences within the Saami family.

sme-sma.




\section{Implementation}
%@fran
% [Pipeline image]

\subsection{Analysis}

\subsubsection{Transfer}


\textbf{Bilingual dictionary}


The bilingual dictionary
general dictionary
general words
special domain

\item problems caused by this model: ex from two terms to one term to two terms

Example: nob-1,2   sme-x   sma-1,2

Little nob-sma lexical resources, and total lack of sme-sma, we have to find words by comparing translated texts




\textbf{Rules}

\item larger syntactic differences between SL an TL than usually done in Apertium?

\subsubsection{Generation}

\section{Evaluation}
%@ lene,fran
\item translation nob-sma which actually is nob-sme-sma (should somehow be in the title?) 
\item the evaluators evaluate nob-sma

\subsection{Statistics}
%@fran
\begin{table}
  \begin{center}
    \begin{tabular}{|l|r|}
      \hline
      Bilingual dictionary ({\tt sme}$\rightarrow${\tt sma}) & 15,204 \\ % 10776 proper names
      Transfer rules ({\tt sme}$\rightarrow${\tt sma}) & 57 \\
      \hline
    \end{tabular}
    \label{table:transfer}
    \caption{Number of bilingual dictionary entries and transfer rules}
  \end{center}
\end{table}

\begin{table}
  \begin{center}
    \begin{tabular}{|l|r|}
      \hline
      \textbf{Corpus} & \textbf{Coverage}  \\
      \hline
      Schoolbooks     & 0.0 $\pm$ 0.0 \\
      General         & 0.0 $\pm$ 0.0 \\
      \hline
    \end{tabular}
    \label{table:coverage}
    \caption{Vocabulary coverage of the system}
  \end{center}
\end{table}

\subsection{Quantitative}
%@lene
\subsection{Qualitative}
%@lene

\subsection{User feedback}
%@lene

\section{Future work}
%@trond
\section{Conclusions}
%@trond
\item reaction from sma linguist in the normgiving organ: she sees the possibility of using MT as a help for discussing terminology and getting the best ones into use
\item the language socity is used to the impact from the mojority languages, a new controversial(?) thought is the linguistic impact from another minority lanugage 


\section*{Acknowledgements}

\bibliographystyle{apalike}
\bibliography{paper}


\end{document}
